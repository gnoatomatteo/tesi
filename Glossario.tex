\appendix
\newpage
%**************************************************************



%**************************************************************
\chapter{Glossario}

\subsubsection{A}
\begin{itemize}
	\item \textbf{Adwords (Google)}: è un servizio online che permette di inserire spazi pubblicitari all'interno delle pagine di ricerca di Google.
	\item \textbf{Agile}: è un metodo di sviluppo del software che propone un approccio meno strutturato e focalizzato sull'obiettivo di consegnare al cliente, in tempi brevi e frequentemente, software funzionante e di qualità.
\end{itemize}

\subsubsection{C}
\begin{itemize}
	\item \textbf{Cloud}: con il termine si indica un paradigma di erogazione di risorse informatiche, come l'archiviazione, l'elaborazione o la trasmissione di dati.
\end{itemize}

\subsubsection{F}
\begin{itemize}
	\item \textbf{Framework}: è un'architettura logica di supporto su cui un software può essere progettato o realizzato, spesso facilitandone lo sviluppo da parte del programmatore.
\end{itemize}

\subsubsection{I}
\begin{itemize}
	\item \textbf{IDE}: \textit{Integrate Development Environment}, è un software che, in fase di programmazione, aiuta i programmatori nello sviluppo del codice sorgente di un programma.
	\item \textbf{IntelliSense}: è una tecnologia che permette il completamento automatico e altri aiuti forniti al programmatore durante la stesura del codice.
\end{itemize}

\subsubsection{L}
\begin{itemize}
	\item \textbf{Linux}: è una famiglia di sistemi operativi di tipo \textit{Unix-like}, rilasciati sotto varie possibili distribuzioni, aventi la caratteristica comune di utilizzare come nucleo il kernel Linux
\end{itemize}

\subsubsection{O}
\begin{itemize}
	\item textbf{OOPSLA}: conferenza annuale che copre argomenti legati alla programmazione orientata agli oggetti di sistemi di programmazione, linguaggi e applicazioni.
\end{itemize}

\subsubsection{S}
\begin{itemize}
	\item \textbf{Stakeholders}: è un soggetto (portatore di interesse) influente nei confronti di un'iniziativa economica, che sia un'azienda o un progetto.
\end{itemize}

\subsubsection{U}
\begin{itemize}
	\item \textbf{UI}: \textit{User Interface}, è ciò che si frappone tra una macchina e un utente, consentendo l'interazione tra i due.
	\item \textbf{Unix}: sistema operativo portabile per computer sviluppato da Ken Thompson e Dennis Ritchie.
\end{itemize}

