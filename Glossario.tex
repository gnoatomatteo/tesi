\appendix
\newpage
%**************************************************************



%**************************************************************
\chapter{Glossario}

\subsubsection{A}
\begin{itemize}
	\item \textbf{Adwords (Google)}: è un servizio online che permette di inserire spazi pubblicitari all'interno delle pagine di ricerca di Google.
	\item \textbf{Agile}: è un metodo di sviluppo del software che propone un approccio meno strutturato e focalizzato sull'obiettivo di consegnare al cliente, in tempi brevi e frequentemente, software funzionante e di qualità.
	\item \textbf{Android}: È un sistema embedded progettato principalmente per smartphone e tablet, con interfacce utente specializzate per televisori (Android TV), automobili (Android Auto), orologi da polso (Android Wear), occhiali (Google Glass), e altri.
	\item \textbf{App}: è la classe principale di un'applicazione sviluppata con Xamarin.Forms, ha il compito di stabilire il comportamento di inizializzazione dell'applicativo e si preoccupa di istanziare le eventuale classi Singleton.
	\item \textbf{Application programming interface}: si indica ogni insieme di procedure disponibili al programmatore, di solito raggruppate a formare un set di strumenti specifici per l'espletamento di un determinato compito all'interno di un certo programma. Spesso con tale termine si intendono le librerie software disponibili in un certo linguaggio di programmazione.
	\item \textbf{API}: vedi \textit{Application programming interface}.
	\item \textbf{Assembly}: è una classe fornita dal framework Xamarin.Forms, permette l'accesso ai dati delle risorse condivise e l'accesso ai dati esposti dalle piattaforme specifiche.
	\item \textbf{Attore}: rappresenta un ruolo coperto da un certo insieme di entità interagenti col sistema (inclusi utenti umani, altri sistemi software, dispositivi hardware e così via). Un ruolo corrisponde a una certa famiglia di interazioni correlate che l'attore intraprende col sistema.
\end{itemize}

\subsubsection{C}
\begin{itemize}
	\item \textbf{Cloud}: con il termine si indica un paradigma di erogazione di risorse informatiche, come l'archiviazione, l'elaborazione o la trasmissione di dati.
	\item \textbf{Cross-Platform}: In informatica il termine multipiattaforma può essere riferito ad un linguaggio di programmazione, ad un'applicazione software o ad un dispositivo hardware che funziona su più di un sistema o appunto, piattaforma (es. Unix/Linux, Windows e Macintosh). Esempi di linguaggi multipiattaforma sono: C, C++, Java, JavaScript, Perl, PHP, Python, Tcl, Erlang e REALbasic.
\end{itemize}

\subsubsection{D}
\begin{itemize}
	\item \textbf{Data binding}: è una tecnica che lega la fonte dei dati dal produttore al consumatore e li sincronizza. Questa tecnica è utile per la separazione di dipendenza tra componenti grafiche e modelli di dati.
	\item \textbf{Database}: indica un insieme organizzato di dati.
	\item \textbf{Debug}:  indica l'attività che consiste nell'individuazione da parte del programmatore della porzione di software affetta da errore (bug) rilevata nei software a seguito dell'utilizzo del programma.
	\item \textbf{Design pattern}: è un concetto che può essere definito "una soluzione progettuale generale ad un problema ricorrente". Si tratta di una descrizione o modello logico da applicare per la risoluzione di un problema che può presentarsi in diverse situazioni durante le fasi di progettazione e sviluppo del software, ancor prima della definizione dell'algoritmo risolutivo della parte computazionale. È un approccio spesso efficace nel contenere o ridurre il debito tecnico.
	\item \textbf{Dependency service}: è una tecnica utilizzata da Xamarin.Forms per accedere alle funzioni delle librerie native della piattaforma, permettendo, al codice condiviso, di non avere dipendenze con il codice nativo, risolvendo la dipendenza a \textit{runtime}.
	\item \textbf{Domain model}: rappresenta il modello di dati, utilizzato nei pattern architetturali per garantire la \textit{separation of concerns}.
	\item \textbf{DLL}: è una libreria software che viene caricata dinamicamente in fase di esecuzione, invece di essere collegata staticamente a un eseguibile in fase di compilazione.
\end{itemize}

\subsubsection{F}
\begin{itemize}
	\item \textbf{Framework}: è un'architettura logica di supporto su cui un software può essere progettato o realizzato, spesso facilitandone lo sviluppo da parte del programmatore.
\end{itemize}

\subsubsection{G}
\begin{itemize}
	\item \textbf{GalaSoft}: è una community fondata da Laurent Bugnion con lo scopo di fornire framework per facilitare l'utilizzo di design pattern specifici, fornendo una struttura pronta.
\end{itemize}

\subsubsection{H}
\begin{itemize}
	\item \textbf{Hardware}: indica la parte fisica di un computer, ovvero tutte quelle parti elettroniche, elettriche, meccaniche, magnetiche, ottiche che ne consentono il funzionamento. L'insieme di tali componenti è anche detto strumentario o componentistica
\end{itemize}

\subsubsection{I}
\begin{itemize}
	\item \textbf{IDE}: \textit{Integrate Development Environment}, è un software che, in fase di programmazione, aiuta i programmatori nello sviluppo del codice sorgente di un programma.
	\item \textbf{Input}:  è un termine inglese con significato di "immettere" che in campo informatico definisce una sequenza di dati o informazioni, immessi per mezzo di una "periferica detta appunto di input" e successivamente elaborati. Il termine, approdato in Italia con la prima informatica degli anni sessanta indicava al contempo i dati di entrata e i supporti che li contenevano.
	\item \textbf{IntelliSense}: è una tecnologia che permette il completamento automatico e altri aiuti forniti al programmatore durante la stesura del codice.
	\item \textbf{Interfaccia}:  si intende un tipo analogo alla classe, ma vincolato a non definire l'implementazione dei propri metodi. Le interfacce rivestono un ruolo importante nell'ambito dell'ereditarietà tra tipi.
\end{itemize}

\subsubsection{J}
\begin{itemize}
	\item \textbf{Json}: è un formato adatto all'interscambio di dati fra applicazioni client-server.
\end{itemize}

\subsubsection{R}
\begin{itemize}
	\item \textbf{Refactoring}:  è una "tecnica strutturata per modificare la struttura interna di porzioni di codice senza modificarne il comportamento esterno", applicata per migliorare alcune caratteristiche non funzionali del software. I vantaggi che il refactoring persegue riguardano in genere un miglioramento della leggibilità della manutenibilità, della riusabilità e dell'estendibilità del codice e la riduzione della sua complessità, eventualmente attraverso l'introduzione a posteriori di design pattern.[2] Il refactoring è un elemento importante delle principali metodologie emergenti di sviluppo del software (soprattutto object-oriented): per esempio delle metodologie agili, dell'extreme programming, e del test driven development.
	\item \textbf{REST}: è un tipo di architettura software per i sistemi di ipertesto distribuiti come il World Wide Web. L'espressione "representational state transfer" e il suo acronimo "REST" furono introdotti nel 2000 nella tesi di dottorato di Roy Fielding, uno dei principali autori delle specifiche dell'Hypertext Transfer Protocol (HTTP), e vennero rapidamente adottati dalla comunità di sviluppatori su Internet.
	\item \textbf{Runtime}: indica il momento in cui un programma per computer viene eseguito, in contrapposizione ad altre fasi del ciclo di vita del software.
\end{itemize}

\subsubsection{L}
\begin{itemize}
	\item \textbf{Layout}: l'impaginazione e la struttura grafica di un sito web o di un documento (come quelli generati da un programma di videoscrittura).
	\item \textbf{Linux}: è una famiglia di sistemi operativi di tipo \textit{Unix-like}, rilasciati sotto varie possibili distribuzioni, aventi la caratteristica comune di utilizzare come nucleo il kernel Linux.
	\item \textbf{Look-and-feel}: l'espressione inglese look and feel viene usata per descrivere le caratteristiche percepite dall'utente di una interfaccia grafica, sia in termini di apparenza visiva (il look) che di modalità di interazione (il feel). Ogni sistema operativo dotato di interfaccia grafica ha un proprio look and feel distintivo, che viene in genere ereditato dalle applicazioni sviluppate per quel sistema; questo favorisce l'usabilità del software, poiché l'utente che impara a usare l'interfaccia di una determinata applicazione è in grado successivamente di riutilizzare la conoscenza acquisita anche nell'uso di altre applicazioni dotate di uguale look and feel.
\end{itemize}

\subsubsection{M}
\begin{itemize}
	\item \textbf{Markup}: è un insieme di regole che descrivono i meccanismi di rappresentazione (strutturali, semantici o presentazionali) di un testo che, utilizzando convenzioni standardizzate, sono utilizzabili su più supporti. La tecnica di formattazione per mezzo di marcatori (o espressioni codificate) richiede quindi una serie di convenzioni, ovvero appunto di un linguaggio a marcatori di documenti.
	\item \textbf{Microsoft}: è una delle più importanti aziende d'informatica del mondo, la più grande produttrice di software al mondo per fatturato, nonché anche una delle più grandi aziende per capitalizzazione azionaria, superiore ai 415 miliardi di dollari nel 2016. Ha sede a Redmond nello stato di Washington (Stati Uniti).
	\item \textbf{Model-View-ViewModel}:  è un pattern software architetturale o schema di progettazione software. È una variante del pattern "Presentation Model design" di Martin Fowler. Lo MVVM astrae lo stato di "view" (visualizzazione) e il comportamento. Sebbene, dove il modello di "presentazione" astrae una vista (crea un view model ) in una maniera che non dipende da una specifica piattaforma interfaccia utente. Lo MVVM fu sviluppato da Ken Cooper e Ted Peters di Microsoft per semplificare la programmazione a eventi di interfacce utente sfruttando caratteristiche del Windows Presentation Foundation (WPF) (Sistema grafico di Microsoft .NET) e Silverlight (Applicazione internet derivata). Il pattern architetturale fu annunciato per la prima volta nel blog di John Gossman nel 2005.
	\item \textbf{MonoDroid}:  Mono per il sistema operativo Android. Con le associazioni per le API di Android.
	\item \textbf{Mono for Android}: è un progetto open source guidato da Xamarin che permette la creazione di applicazione Android, con il linguaggio C\#, un compilatore e un \textit{Common Language Runtime}. 
	\item \textbf{MonoTouch}: Mono per il touch di iPhone, iPad e iPod. Con attacchi alle API iOS.
	\item \textbf{MVVM}: vedi la voce \textit{Model-View-ViewModel}.
	\item \textbf{MVVMLight}: pacchetto NuGet per l'utilizzo degli strumenti che aiutano la realizzazione di un MVVM distribuito da \textit{GalaSoft}.
\end{itemize}

\subsubsection{N}
\begin{itemize}
	\item \textbf{.Net}: è un progetto all'interno del quale Microsoft ha creato una piattaforma di sviluppo software, .NET, che è una versatile tecnologia di programmazione ad oggetti.
\end{itemize}

\subsubsection{O}
\begin{itemize}
	\item \textbf{OOPSLA}: conferenza annuale che copre argomenti legati alla programmazione orientata agli oggetti di sistemi di programmazione, linguaggi e applicazioni.
\end{itemize}

\subsubsection{P}
\begin{itemize}
	\item \textbf{Parser}: è un processo che analizza un flusso continuo di dati in ingresso (input) (letti per esempio da un file o una tastiera) in modo da determinare la sua struttura grazie ad una data grammatica formale. Un parser è un programma che esegue questo compito.
	\item \textbf{PDF}: formato di file per rappresentare documenti in modo indipendente dall'hardware e dal software.
	\item \textbf{PCL}: è un tipo di progetto Xamarin.Forms che permette la creazione di libreria di classi portabili e indipendenti della piattaforma sottostante.
	\item \textbf{Property}: è un campo con permesso di accesso e modifica, comune nel linguaggio C\# e indispensabile per eseguire il \textit{Data-Binding}.
\end{itemize}

\subsubsection{S}
\begin{itemize}
	\item \textbf{Server}: è un componente o sottosistema informatico di elaborazione e gestione del traffico di informazioni che fornisce, a livello logico e fisico, un qualunque tipo di servizio ad altre componenti (tipicamente chiamate clients, cioè clienti) che ne fanno richiesta attraverso una rete di computer, all'interno di un sistema informatico o anche direttamente in locale su un computer.
	\item \textbf{Shared Project}: è untipo di progetto Xamarin.Forms che permette di condividere il codice tra diverse piattaforme.
	\item \textbf{Silverlight}:  è un ambiente di runtime sviluppato da Microsoft per piattaforme Windows e Mac che consente di visualizzare, all'interno del browser, Rich Internet application, ovvero applicazioni multimediali ad alta interattività. Per le altre piattaforme, come quelle basate sul kernel Linux, è disponibile da parte di Novell un'implementazione opensource chiamata Moonlight, del cui sviluppo si occupava il progetto Mono.
	\item \textbf{Sistema operativo}: è un insieme di componenti software, che rende operativi (da cui il nome) computer, apparati e dispositivi informatici.
	\item \textbf{SQLite}:  è una libreria software scritta in linguaggio C che implementa un DBMS SQL di tipo ACID incorporabile all'interno di applicazioni. Il suo creatore, D. Richard Hipp, lo ha rilasciato nel pubblico dominio, rendendolo utilizzabile quindi senza alcuna restrizione. Permette di creare una base di dati (comprese tabelle, query, form, report) incorporata in un unico file, come nel caso dei moduli Access di Microsoft Office e Base di OpenOffice.org e Libre Office; analogamente a prodotti specifici come Paradox o Filemaker.
	\item \textbf{Stakeholders}: è un soggetto (portatore di interesse) influente nei confronti di un'iniziativa economica, che sia un'azienda o un progetto.
	
\end{itemize}

\subsubsection{T}
\begin{itemize}
	\item \textbf{Two-Way Data-Binding}: esegue la sincronizzazione dei dati in \textit{Data-Binding} in doppia direzione.
\end{itemize}

\subsubsection{X}
\begin{itemize}
	\item \textbf{Xamarin}:  è un'azienda produttrice di software statunitense con sede a San Francisco, California fondata nel maggio 2011 dagli ingegneri di Mono, Mono per Android e MonoTouch che sono implementazioni multipiattaforma di Common Language Infrastructure (Infrastrutture di linguaggio comune) e di Common Language Specifications (spesso chiamato Microsoft .NET). Con un codice condiviso basato su C\#, gli sviluppatori possono usare gli strumenti Xamarin per scrivere applicazioni native Android, iOS e Windows con interfacce utenti native e condividere il codice su diverse piattaforme. Xamarin ha oltre 1 milione di sviluppatori in più di 120 paesi nel mondo (Maggio 2015)
	\item \textbf{Xamarin.Forms}: è un toolkit di interfaccia utente cross-platform che consente agli sviluppatori di condividere UI tra Android, iOS e Windows Phone, 
	\item \textbf{XAML}:  è un linguaggio di markup basato su XML, utilizzato per descrivere l'interfaccia grafica delle applicazioni basate sulla libreria Windows Presentation Foundation.
	\item \textbf{XML}: è un metalinguaggio per la definizione di linguaggi di markup, ovvero un linguaggio marcatore basato su un meccanismo sintattico che consente di definire e controllare il significato degli elementi contenuti in un documento o in un testo.
	\item \textbf{XSLT}: è il linguaggio di trasformazione dell'XML, diventato uno standard web con una direttiva (Recommendation) W3C del 16 novembre 1999. L'obiettivo principale per cui l'XSLT è stato creato è rendere possibile la trasformazione di un documento XML in un altro documento. Deriva direttamente dal linguaggio XSL, infatti i file di questo formato sono essenzialmente file di testo, contengono elementi ed attributi ed hanno l'estensione ".xsl".
\end{itemize}

\subsubsection{V}
\begin{itemize}
	\item \textbf{View}: è la componente grafica del design pattern \textit{MVVM}.
	\item \textbf{ViewModel}: è la componente ViewModel del design pattern \textit{MVVM}. Questa componente ha la responsabilità di gestire i dati per proporli correttamente alle viste.
	\item \textbf{Visual Studio}: è un ambiente di sviluppo integrato (Integrated development environment o IDE) sviluppato da Microsoft, che supporta attualmente diversi tipi di linguaggio, quali C, C++, C\#, F\#, Visual Basic .Net, Html e JavaScript, e che permette la realizzazione di applicazioni, siti web, applicazioni web e servizi web.
\end{itemize}

\subsubsection{W}
\begin{itemize}
	\item \textbf{Windows}: è una famiglia di ambienti operativi e sistemi operativi dedicati ai personal computer, alle workstation, ai server e agli smartphone. Il sistema operativo si chiama così per via della sua interfaccia di programmazione di un'applicazione a finestre.
	\item \textbf{Windows Phone}: è una famiglia di sistemi operativi per smartphone di Microsoft, presentata per la prima volta al Mobile World Congress il 15 febbraio 2010.
	\item \textbf{WPF}: è una libreria di classi del Framework .NET proprietarie Microsoft (introdotta con la versione 3.0) per lo sviluppo dell'interfaccia grafica delle applicazioni in ambienti Windows.
	
	
\end{itemize}

\subsubsection{U}
\begin{itemize}
	\item \textbf{UI}: \textit{User Interface}, è ciò che si frappone tra una macchina e un utente, consentendo l'interazione tra i due.
	\item \textbf{Unix}: sistema operativo portabile per computer sviluppato da Ken Thompson e Dennis Ritchie.
\end{itemize}

