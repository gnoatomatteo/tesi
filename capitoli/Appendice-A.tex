\newpage
%**************************************************************
\chapter{Requisiti}
\label{cap:Requisiti}


	\begin{longtable}{||p{1,5cm} p{10.55cm}||} 
		\hline
		\textbf{ID} & \textbf{Requisito} \\\toprule 
		\hline\hline
		1		 & L’utente può configurare il proprio profilo nell’applicazione  \\ 
		1.1		 & L’utente può inserire il proprio nome \\
		1.2 	 & L’utente può inserire il proprio cognome \\
		1.3		 & L’utente può inserire il proprio indirizzo \\
		1.3.1	 & L’utente può inserire la propria via \\
		1.3.2	 & L’utente può inserire il proprio cap. \\
		1.3.3	 & L’utente può inserire la propria città \\
		1.4		 & L’utente può inserire il numero di cellulare \\
		1.5		 & L’utente può inserire il numero di telefono \\
		1.6		 & L’utente può inserire la propria mail \\
		1.7		 & L’utente può inserire le proprie qualifiche \\
		\hline
		2 & L’utente può identificarsi all’interno dell’applicazione con le credenziali che gli sono state fornite  \\
		2.1 & L’utente può inserire il proprio nome utente \\
		2.2 & L’utente può inserire la propria password associata al proprio nome utente \\
		2.3 & L’utente può confermare la propria autenticazione \\
		\hline
		3 & L’utente può inserire un nuovo cliente \\
		3.1 & L’utente può inserire il nome del nuovo cliente\\
		3.2 & L’utente può inserire l’indirizzo del nuovo cliente\\
		3.2.1 & L’utente può inserire la via del nuovo cliente\\
		3.2.2 & L’utente può inserire il numero civico del nuovo cliente\\
		3.2.3 & L’utente può inserire la città del nuovo cliente\\
		3.2.4 & L’utente può inserire il cap. del nuovo cliente\\
		3.2.5 & L’utente può inserire la provincia del nuovo cliente\\
		3.3 & L’utente può confermare l’inserimento del cliente\\
		\hline
		4 & L’utente identificato non può aggiungere un nuovo cliente e aggiungere articoli perché li scarica da un servizio predisposto \\
		\hline
		5 &  L’utente e l’utente identificato possono ricercare i rapportini\\
		5.1 & L’utente e l’utente identificato possono inserire una stringa di ricerca\\
		5.2 & L’utente e l’utente identificato possono confermare la ricerca\\
		5.3 & L’utente e l’utente identificato possono scegliere una ricerca rapida proposta dal sistema\\
		5.3.1 & L’utente e l’utente identificato possono scegliere la ricerca di tutti i rapportini\\
		5.3.2 & L’utente e l’utente identificato possono scegliere la ricerca di tutti i rapportini non conclusi (default quando viene visualizzata la home page)\\
		5.3.3 & L’utente e l’utente identificato possono scegliere la ricerca di tutti i rapportini creati nel mese corrente\\
		5.3.4 & L’utente e l’utente identificato possono scegliere la ricerca di tutti i rapportini creati nella giornata corrente\\ 
		\hline
		6 & L’utente e l’utente identificato possono visualizzare e modificare i rapportini generati in precedenza in stato di “non concluso” che verranno visualizzati in una lista come ricerca standard (quindi senza che l’utente effettui una reale ricerca)\\
		\hline
		7 & L’utente può creare un nuovo rapportino\\
		7.1 & L’utente può selezionare la propria qualifica nella mansione\\
		7.2 & L’utente può selezionare un cliente\\
		7.3 & L’utente può inserire un cliente se non è già presente, quest’ultimo verrà salvato come nuovo cliente\\
		7.4 & L’utente può inserire una locazione diversa dalla sede del cliente\\
		7.4.1 & L’utente può inserire la ragione sociale del cliente\\
		7.4.2 & L’utente può inserire l’indirizzo del cliente\\
		7.4.2.1 & L’utente può inserire la via del cliente\\
		7.4.2.2 & L’utente può inserire il numero civico del cliente\\
		7.4.2.3 & L’utente può inserire il cap. del cliente \\
		7.4.2.4 & L’utente può inserire la città del cliente\\
		7.4.2.5 & L’utente può inserire la provincia del cliente\\
		7.5 & L’utente può inserire l’orario mattutino\\
		7.5.1 & L’utente può inserire l’orario di arrivo mattutino\\
		7.5.2 & L’utente può inserire l’orario di partenza mattutino\\
		7.6 & L’utente può inserire l’orario pomeridiano\\
		7.6.1 & L’utente può inserire l’orario di arrivo pomeridiano\\
		7.6.2 & L’utente può inserire l’orario di partenza pomeridiano\\
		7.7 & L’utente può inserire le ore di viaggio\\
		7.8 & L’utente può inserire i chilometri percorsi\\
		7.9 & L’utente può inserire il costo dell’autostrada\\
		7.10 & L’utente può inserire il costo del parcheggio\\
		7.11 & L’utente può inserire il costo dei pasti\\
		7.12 & L’utente può inserire il costo di pernottamento\\
		7.13 & L’utente può inserire un articolo\\
		7.13.1 & L’utente può inserire una descrizione dell’articolo\\
		7.13.2 & L’utente può inserire il tempo parziale di un articolo\\
		7.13.3 & L’utente può inserire il tempo totale di un articolo\\
		7.14 & L’utente può confermare la creazione del rapportino\\
		\hline
		8 & L’utente identificato può creare un nuovo rapportino\\
		8.1 & L’utente identificato può selezionare la propria qualifica nella mansione\\
		8.2 & L’utente identificato può selezionare un cliente\\
		8.3 & L’utente identificato può inserire un cliente se non è presente nella sua anagrafica\\
		8.4 & L’utente identificato può inserire una locazione diversa dalla sede del cliente\\
		8.4.1 & L’utente identificato può inserire la ragione sociale del cliente\\
		8.4.2 & L’utente identificato può inserire l’indirizzo del cliente\\
		8.4.2.1 & L’utente identificato può inserire la via del cliente\\
		8.4.2.2 & L’utente identificato può inserire il numero civico del cliente\\
		8.4.2.3 & L’utente identificato può inserire il cap. del cliente \\
		8.4.2.4 & L’utente identificato può inserire la città del cliente\\
		8.4.2.5 & L’utente identificato può inserire la provincia del cliente\\
		8.5 & L’utente identificato può inserire l’orario mattutino\\
		8.6 & L’utente identificato può inserire l’orario pomeridiano\\
		8.7 & L’utente identificato può selezionare le ore di viaggio\\
		8.8 & L’utente identificato può inserire i chilometri percorsi\\
		8.9 & L’utente identificato può inserire il costo dell’autostrada\\
		8.10 & L’utente identificato può inserire il costo del parcheggio\\
		8.11 & L’utente identificato può inserire il costo dei pasti\\
		8.12 & L’utente identificato può inserire il costo di pernottamento\\
		8.13 & L’utente identificato può selezionare un articolo\\
		8.14 & L’utente identificato può confermare la creazione del rapportino\\
		\hline
		9 & L’utente può modificare il rapportino trovato se non è in stato ‘concluso’\\
		9.1 & L’utente può cambiare il cliente del rapportino\\
		9.2 & L’utente può cambiare l’orario mattutino del rapportino\\
		9.2.1 & L’utente può cambiare l’ora di arrivo mattutino\\
		9.2.2 & L’utente può cambiare l’ora di partenza mattutina\\
		9.3 & L’utente può cambiare l’orario pomeridiano del rapportino\\
		9.3.1 & L’utente può cambiare l’ora di arrivo pomeridiana\\
		9.3.2 & L’utente può cambiare l’ora di partenza pomeridiana\\
		9.4 & L’utente può cambiare la locazione dell’intervento\\
		9.4.1 & L’utente può cambiare la via del cliente\\
		9.4.2 & L’utente può cambiare il numero civico del cliente \\
		9.4.3 & L’utente può cambiare il cap. del cliente\\
		9.4.4 & L’utente può cambiare la città del cliente\\
		9.4.5 & L’utente può cambiare la provincia del cliente\\
		9.5 & L’utente può cambiare le ore di viaggio\\
		9.6 & L’utente può cambiare i chilometri percorsi\\
		9.7 & L’utente può cambiare il costo dell’autostrada\\
		9.8 & L’utente può cambiare il costo del parcheggio\\
		9.9 & L’utente può cambiare il costo dei pasti\\
		9.10 & L’utente può cambiare il costo di pernottamento\\
		9.11 & L’utente può cambiare un articolo inserito nel rapportino\\
		9.11.1 & L’utente può scegliere un nuovo articolo da sostituire a quello esistente\\ 
		9.12 & L’utente può modificare un articolo inserito nel rapportino\\
		9.12.1 & L’utente può modificare la descrizione dell’articolo\\
		9.12.2 & L’utente può modificare il tempo parziale dell’articolo\\
		9.13 & L’utente può inserire un nuovo articolo nel rapportino che sta modificando\\
		9.14 & L’utente può rimuovere un articolo presente nel rapportino che sta modificando\\
		9.15 & L’utente può confermare le modifiche apportate\\
		9.16 & L’utente può annullare le modifiche apportate\\	
		\hline
		10 & L’utente identificato può modificare il rapportino trovato se non è in stato ‘concluso’\\
		10.1 & L’utente identificato può cambiare l’orario mattutino del rapportino\\
		10.1.1 & L’utente identificato può cambiare l’ora di arrivo mattutino\\
		10.1.2 & L’utente identificato può cambiare l’ora di partenza mattutina\\
		10.2 & L’utente identificato può cambiare l’orario pomeridiano del rapportino\\
		10.2.1 & L’utente identificato può cambiare l’ora di arrivo pomeridiana\\
		10.2.2 & L’utente identificato può cambiare l’ora di partenza pomeridiana\\
		10.3 & L’utente identificato può cambiare le ore di viaggio standard\\
		10.3.1 & L’utente identificato può impostare le ore di viaggio standard\\
		10.3.2 & L’utente identificato può impostare l’inserimento manuale delle ore di viaggio\\
		10.4 & L’utente identificato può cambiare i chilometri percorsi\\
		10.4.1 & L’utente identificato può impostare i chilometri trascorsi standard\\
		10.4.2 & L’utente identificato può impostare l’inserimento manuale dei chilometri trascorsi\\
		10.5 & L’utente identificato può cambiare il costo dell’autostrada\\
		10.6 & L’utente identificato può cambiare il costo del parcheggio\\
		10.7 & L’utente identificato può cambiare il costo dei pasti\\
		10.8 & L’utente identificato può cambiare il costo di pernottamento\\
		10.9 & L’utente identificato può cambiare un articolo inserito nel rapportino\\
		10.9.1 & L’utente identificato può scegliere un nuovo articolo da sostituire a quello esistente\\
		10.10 & L’utente identificato può modificare un articolo inserito nel rapportino\\
		10.10.1 & L’utente identificato può modificare la descrizione dell’articolo\\
		10.10.2 & L’utente identificato può modificare il tempo parziale dell’articolo\\
		10.10.3 & L’utente identificato può modificare il tempo parziale dell’articolo\\
		10.11 & L’utente identificato può inserire un nuovo articolo nel rapportino che sta modificando\\
		10.12 & L’utente identificato può rimuovere un articolo presente nel rapportino che sta modificando\\
		10.13 & L’utente identificato può confermare le modifiche apportate\\
		10.14 & L’utente identificato può annullare le modifiche apportate\\
		\hline
		11 & L’utente può eliminare il rapportino trovato\\
		11.1 & L’utente può confermare la rimozione del rapportino\\
		\hline
		12 & L’utente identificato può eliminare un rapportino da lui creato, ma lo rimuove solo a livello logico nel database locale (resta dunque in coda per l’invio al server o nel database remoto)\\
		12.1 & L’utente identificato può confermare la rimozione del rapportino\\
		\hline
		13 & L’utente può inviare un rapportino, via mail o con applicazioni di messaggistica istantanea\\
		\hline
		14 & L’utente identificato può inviare il rapportino se il rapportino è in stato di ‘concluso’, via mail o con applicazioni di messaggistica istantanea\\
		\hline
		15 & L’utente e l’utente identificato possono visualizzare il rapportino trovato\\
		15.1 & L’utente e l’utente identificato possono visualizzare le informazioni del rapportino \\
		\hline
		16 & L’utente può firmare il rapportino trovato se non è in stato di ‘concluso’, aprendolo con un’applicazione esterna \\
		\hline
		17 & L’utente può segnalare come ‘concluso’ un rapportino ‘non concluso’\\
		\hline
		18 & Il cliente può firmare un documento, aprendolo con una applicazione esterna\\
		\hline
		19 & L’utente identificato può forzare la sincronizzazione dell’applicazione\\
		\hline
		20 & L’utente identificato, per motivi di sicurezza, può bloccare l’applicazione fino all’inserimento delle password\\
		\hline
		21 & L’utente identificato può sbloccare l’applicazione inserendo la propria password\\
		\hline
		22 & L’utente identificato può richiedere il recupero della propria password\\
		\hline
		23 & L’utente può visualizzare le informazioni sull’ applicazione\\
		\hline
		24 & L’applicazione ricerca i rapportini non conclusi in automatico e li propone all’utente quando non ha eseguito una ricerca più specifica\\
		\hline
		25 & L’applicazione di occupa di gestire le liste di articoli che l’utente identificato può inserire nei rapportini\\
		\hline
		26 & L’applicazione deve sincronizzare il proprio database locale con un servizio remoto\\
		26.1 & L’applicazione deve sincronizzarsi da un servizio predisposto\\
		26.1.1 & L’applicazione deve scaricare i dati da un servizio predisposto\\
		26.1.2 & L’applicazione deve trasmettere il documento ad un server predisposto a riceverlo\\
		26.1.3 & L’applicazione sincronizza i clienti dal servizio predisposto \\
		26.1.4 & L’applicazione sincronizza i rapportini dal servizio predisposto\\
		26.1.5 & L’applicazione sincronizza gli articoli disponibili dal servizio predisposto\\
		\hline
		27 & L’applicazione permette l’invio di un documento\\
		27.1 & L’applicazione permette l’invio di un documento tramite e-mail\\
		27.1.1 & L’applicazione accede all’applicazione predefinita per l’invio della mail del dispositivo preparando una mail con allegato\\
		27.2 & L’applicazione permette l’invio di un documento tramite servizi di messaggistica istantanea\\
		27.2.1 & L’applicazione permette l’invio di un documento tramite whatsapp\\
		27.2.2 & L’applicazione permette l’invio di un documento tramite telegram\\
		\hline
		28 & L’applicazione mette a disposizione un tutorial per guidare l’utente all’utilizzo dell’applicazione\\
		\hline
\end{longtable}