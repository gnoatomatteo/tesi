\newpage
\chapter{Conclusioni}
\label{cap:conclusioni}

\section{Test e giudizi sul prodotto}
Il prototipo dell'applicativo sviluppato durante l'esperienza di stage è stato sottoposto ad una serie di test sia da parte mia, sia dal tutor aziendale, per verificare le funzionalità e la corrispondenza ai requisiti definiti in fase di analisi.
\section{Problematiche riscontrate}
Durante lo sviluppo del'applicativo ho subito numerosi rallentamenti a causa della ricerca della creazione del rapportino a partire da un layout scambiabile. Inizialmente avevo testato le funzionalità fornite da diverse librerie, ma nessuna forniva ciò che realmente interessava all'azienda. L'uso di \textit{XSLT\ped{G}} è stato decretato in ritardo da come stabilito nel piano, questo ha comportato un ritardo sull'inizio dell'analisi, e di conseguenza di tutte le attività successive.  
\section{Esperienza di stage}
Nell'accingermi alla scelta dello stage, ho puntato sopratutto ad un'esperienza che fosse il più possibile paragonabile ad un effettivo impiego lavorativo; contando di essere inserito e trattato, a tutti gli effetti, come un impiegato dell'azienda ospitante, con lo scopo di ottenere una proficua e valida esperienza formativa.
\\
A conclusione dello stage, ritengo che l'esperienza lavorativa compiuta in \asi abbia soddisfatto in pieno le mie aspettative.
\section{Preparazione corso di studi}
Personalmente ritengo che le competenze acquisite durante i tre anni del corso di studi si siano dimostrate adeguate, per un proficuo inserimento nel mondo del lavoro, specializzato nella creazione e commercializzazione di prodotti informatici.

\section{Conoscenze acquisite}
\begin{itemize}
	\item \textbf{Scrum:} durante gli studi universitari non ho mai preso in considerazione di applicare una precisa metodologia per la realizzazione dei vari progetti. Nel corso dello stage, ho potuto invece verificare come Scrum, adottato da \asi, si sia rilevato particolarmente efficiente per gestire le varie fasi di sviluppo del nuovo prodotto. Questa esperienza mi ha permesso di apprendere l'ideologia su cui è basata la creazione di Scrum, ho potuto collaudare il corretto svolgimento di tale metodologia ed ho apprezzato la sua applicazione nei processi aziendali.
	\item \textbf{Sviluppo Mobile:} non avendo mai svolto in precedenza alcun progetto su piattaforme mobili, mi sono sentito particolarmente soddisfatto del risultato ottenuto con lo stage, nel corso del quale ho appreso dettagliatamente come sono gestiti gli applicativi delle varie piattaforme, consentendomi di familiarizzare motlo con lo sviluppo delle applicazioni per dispositivi mobili e di ampliare le mie conoscenze su una disciplina non ancora approfondita nei corsi universitari.
	\item \textbf{Xamarin.Forms:} per sviluppare l'applicativo durante lo stage ho dovuto imparare a conoscere ed utilizzare Xamarin.Forms che si è rilevato uno strumento abbastanza efficace per lo sviluppo \textit{cross-platform\ped{G}}.
\end{itemize}