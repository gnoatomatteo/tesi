\newpage
\chapter{Conclusioni}
\label{cap:conclusioni}

\section{Soddisfazione obiettivi}
Il prototipo dell'applicativo sviluppato durante l'esperienza di stage è stato sottoposto ad una serie di test sia da parte mia, sia dal tutor aziendale, per verificare le funzionalità e la corrispondenza ai requisiti definiti in fase di analisi.
\\
Al termine dello stage, mi è stato chiesto di fare una presentazione dell'applicativo e illustrare le funzionalità ottenute. La presentazione è stata seguita dal tutor aziendale, il committente e i soci dell'azienda, che hanno trovato nel risultato diversi spunti che ritengono interessanti e che intendono approfondire nei progetti futuri, come l'uso di un database locale nel dispositivo mobile e la generazione del \textit{PDF\ped{G}} senza l'utilizzo della rete. 
\section{Problematiche riscontrate}
Durante lo sviluppo del'applicativo ho subito dei rallentamenti a causa della ricerca sulla creazione del rapportino a partire da un layout scambiabile. Inizialmente avevo testato le funzionalità fornite da diverse librerie, ma nessuna forniva ciò che realmente interessava all'azienda. L'uso di \textit{XSLT\ped{G}} è stato decretato dopo aver effettuato delle prove concrete con il pacchetto \textit{NuGet\ped{G}} \textit{Xam.iTextSharp\ped{G}}, perciò in ritardo da come stabilito nel piano, questo ha comportato un ritardo sull'inizio dell'analisi, e di conseguenza di tutte le attività successive.  
\section{Giudizi sul prodotto e test}
Per eseguire dei test pratici è stato installato l'applicativo in un dispositivo fisico, fornito dall'azienda, dotato di sistema operativo \textit{Android\ped{G}}, gli altri sistemi sono stati testati soltanto mediante simulatori, per mancanza di dispositivi fisici a disposizione. \\
Lo scopo era, infatti, quello di verificare la piena compatibilità dell'applicativo con le piattaforme, dimostrando anche l'adattabilità dell'interfaccia grafica fornita da \textit{Xamarin\ped{G}} per dispositivi con sistemi diversi.
\\
Anche se i requisiti non sono stati tutti soddisfatti, il risultato è stato positivo, in quanto l'applicazione durante la presentazione è risultata stabile e funzionante per tutte le funzioni presentate.
\\
Durante la presentazione dell'applicativo, inoltre, è stata presentata la metodologia per la generazione del \textit{PDF\ped{G}} e esposte le prove eseguite con le diverse librerie, visto il peso che ha avuto la ricerca della migliore soluzione da adottare durante il periodo di stage.
\section{Considerazioni sul prodotto}
Personalmente mi ritengo soddisfatto del prodotto sviluppato, di fatto l'applicativo è risultato solido alle prove effettuate e esegue correttamente le funzionalità implementate.
\\
Avrei indubbiamente preferito che fosse dotato di tutte le caratteristiche richieste  in origine dall'azienda, ma, a causa dei ritardi subiti, ho preferito concentrarmi sull'obiettivo di mantenere una buona efficienza delle funzionalità indispensabili, piuttosto che all'aggiunta di funzioni aggiuntive su un prodotto instabile. Tale scelta è stata infatti apprezzata dall'azienda.
\section{Esperienza di stage}
Nell'accingermi alla scelta dello stage, ho puntato sopratutto ad un'esperienza che fosse il più possibile paragonabile ad un effettivo impiego lavorativo; contando di essere inserito e trattato, a tutti gli effetti, come un impiegato dell'azienda ospitante, con lo scopo di ottenere una proficua e valida esperienza formativa.
\\
Fin dal primo giorno, sono stato inserito come un qualsiasi impiegato, con l'impegno di essere regolarmente presente in azienda, di rispettare gli stessi orari di lavoro degli altri membri del team di sviluppo e di adattarmi alle consuetudini locali, in modo da procurarmi un esperienza lavorativa quanto più possibile concreta e proficua.
\\
Ho trovato nel personale dell'azienda una grande disponibilità; il tutor aziendale, i membri del team di sviluppo e tutti gli altri impiegati mi hanno messo a mio agio fin dai primi giorni, consentendomi di lavorare in un ambiente accogliente e sereno, ma comunque serio e professionale.
\\
Come avevo compreso fin dal colloquio iniziale con il tutor, sia lui stesso, sia il committente del progetto si sono sempre dimostrati disponibili a discutere l'andamento dello stage, fornendo chiarimenti sulle funzioni da implementare nel prodotto e controllando di frequentare gli incrementi ottenuti nello sviluppo del prototipo.
\\
Il team di sviluppo ha offerto il proprio supporto tecnico ogni volta che mi poteva essere utile, fornendo i dettagli sulla gestione delle funzionalità dell'applicativo.
\\
Ritengo quindi di aver acquisito un'esperienza molto utile e formativa, che mi ha concretamente avvicinato al tipo di lavoro che vorrei svolgere e alle dinamiche della produzione e della commercializzazione dei prodotti così particolari ed innovativi, come sono quelli provenienti dalla creatività abbinata alla tecnologia.
\\ 
A conclusione dello stage, ritengo che l'esperienza lavorativa compiuta in \asi abbia soddisfatto in pieno le mie aspettative.

\section{Preparazione corso di studi}
Personalmente ritengo che le competenze acquisite durante i tre anni del corso di studi si siano dimostrate adeguate, per un proficuo inserimento nel mondo del lavoro, specializzato nella creazione e commercializzazione di prodotti informatici.
\\
Sixuramente mi sarebbe risultato molto utile un approccio, anche di carattere generale, allo sviluppo su piattaforme mobili, considerando che l'impiego dei dispositivi portabili si sta sempre più diffondendo e le aziende, per rimanere competitive, sono stimolate a dotare i loro prodotti software di nuove funzioni che ne consentano un efficacie utilizzo anche in mobilità.
\\
Pur privo di questa specifica preparazione, posso tuttavia confermare che, dall'insegnamento delle materie informatiche, ho potuto comunque ricavare dei metodi utili per potermi avvicinare potenzialmente a qualsiasi linguaggio di programmazione ed ambiente di sviluppo, che mi hanno consentito di apprendere facilmente le tecniche più appropriate per la creazione di un applicativo destinato a dispositivi mobili, ancorché funzionanti con sistemi operativi alquanto diversi tra loro.

\section{Conoscenze acquisite}
\begin{itemize}
	\item \textbf{Scrum:} durante gli studi universitari non ho mai preso in considerazione di applicare una precisa metodologia per la realizzazione dei vari progetti. Nel corso dello stage, ho potuto invece verificare come Scrum, adottato da \asi, si sia rilevato particolarmente efficiente per gestire le varie fasi di sviluppo del nuovo prodotto. Questa esperienza mi ha permesso di apprendere l'ideologia su cui è basata la creazione di Scrum, ho potuto collaudare il corretto svolgimento di tale metodologia ed ho apprezzato la sua applicazione nei processi aziendali.
	\item \textbf{Sviluppo Mobile:} non avendo mai svolto in precedenza alcun progetto su piattaforme mobili, mi sono sentito particolarmente soddisfatto del risultato ottenuto con lo stage, nel corso del quale ho appreso dettagliatamente come sono gestiti gli applicativi delle varie piattaforme, consentendomi di familiarizzare molto con lo sviluppo delle applicazioni per dispositivi mobili e di ampliare le mie conoscenze su una disciplina non ancora approfondita nei corsi universitari.
	\item \textbf{Xamarin.Forms:} per sviluppare l'applicativo durante lo stage ho dovuto imparare a conoscere ed utilizzare Xamarin.Forms che si è rilevato uno strumento abbastanza efficace per lo sviluppo \textit{cross-platform\ped{G}}.
\end{itemize}