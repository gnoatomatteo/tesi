%**************************************************************
% Sommario
%**************************************************************
\cleardoublepage
\phantomsection
\pdfbookmark{Sommario}{Sommario}
\begingroup
\let\clearpage\relax
\let\cleardoublepage\relax
\let\cleardoublepage\relax

\chapter*{Sommario}
Al termine dei tre anni di corso di laurea, gli studenti devono intraprendere uno stage della durata massima di 320 ore, che può essere sia interno all'università, sia esterno in un'azienda.
\\
\\
Io ho preferito la seconda opzione, considerando più formativa un'esperienza di lavoro in un'azienda avviata; ho scelto di svolgere il mio stage in ASI S.r.l., dove mi è stato richiesto di sviluppare un applicativo mobile multi-piattaforma per la gestione e la generazione di rapportini firmabili tramite la firma biometrica, realizzando anche i servizi che serviranno all'applicazione di sincronizzarsi in modo intelligente con un database remoto.

\subsubsection*{Convenzioni tipografiche}
Nella redazione della tesi di laurea ho applicato le seguenti convenzioni:
\begin{itemize}
	\item Le citazioni sono evidenziate in corsivo.
	\item Gli acronimi e i termini di uso non comune sono definiti nel glossario e contrassegnati alla loro prima occorrenza dal pedice \ped{G}. 
\end{itemize} 


%\vfill
%
%\selectlanguage{english}
%\pdfbookmark{Abstract}{Abstract}
%\chapter*{Abstract}
%
%\selectlanguage{italian}

\endgroup			

\vfill

